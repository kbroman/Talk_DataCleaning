\documentclass[aspectratio=169,12pt,t]{beamer}
\usepackage{graphicx}
\setbeameroption{hide notes}
\setbeamertemplate{note page}[plain]
\usepackage{listings}
\usepackage{eepic}

\input{header.tex}

%%%%%%%%%%%%%%%%%%%%%%%%%%%%%%%%%%%%%%%%%%%%%%%%%%%%%%%%%%%%%%%%%%%%%%
% end of header
%%%%%%%%%%%%%%%%%%%%%%%%%%%%%%%%%%%%%%%%%%%%%%%%%%%%%%%%%%%%%%%%%%%%%%

% title info
\title{data cleaning principles}
\author{\vspace*{-10pt} \href{https://kbroman.org}{Karl Broman}}
\institute{Biostatistics \& Medical Informatics, UW{\textendash}Madison}
\date{\href{https://twitter.com/kwbroman}{\tt \scriptsize \color{foreground} @kwbroman}
\\[-4pt]
\href{https://kbroman.org}{\tt \scriptsize \color{foreground} kbroman.org}
\\[-4pt]
\href{https://github.com/kbroman}{\tt \scriptsize \color{foreground} github.com/kbroman}
\\[-4pt]
{\scriptsize \href{https://kbroman.org/Talk_DataCleaning}{\tt kbroman.org/Talk\_DataCleaning}}
}


\begin{document}

% title slide
{
\setbeamertemplate{footline}{} % no page number here
\frame{
  \titlepage

  \note{
    These are slides for a talk for the csv,conf,v6 ({\tt
    https://csvconf.com/}) on May 4-5, 2021.

    Data analysts spend a lot of time organizing and cleaning data,
    but few of us have been trained to do so. Why is that?

    Some say that data cleaning is difficult to generalize. But I
    think there are some general principles. Moreover, I think we have
    an important shared experience in data cleaning that we can
    commiserate about, and through which we can learn from each other.
  }

} }


\begin{frame}[c]{}

\centering
\Large

Tidy data are all alike, \\[3pt]
but every messy dataset  \\[3pt]
is messy in its own way.

\bigskip\bigskip

\hspace{5cm} --
\href{https://r4ds.had.co.nz/tidy-data.html}{Hadley Wickham}



\note{
  Hadley's talking more about data organization than data
  cleanliness.  And his point is that if you make data tidy, it
  simplifies all the downstream analyses.

  But {\hilit is} every messy dataset {\vhilit uniquely} messy?
}
\end{frame}


\begin{frame}[c]{}

\centering
\Large

If I clean up [Medicare] data ...\\[3pt]
does any of the knowledge I gain ...\\[3pt]
apply to the processing of RNA-seq data?


\bigskip\bigskip

\hspace{5cm} --
\href{https://doi.org/10.1080/10618600.2017.1385470}{Roger Peng}



\note{
  In his discussion of David Donoho's paper about data science, Roger
  Peng spoke about how data cleaning is frustratingly difficult to
  generalize.

  But my answer to his question is {\vhilit absolutely!}

  A person with experience cleaning one dataset has important
  experience to draw upon when moving to another dataset even if it's
  of a totally different nature.
}
\end{frame}

\begin{frame}[c]{}

  \figh{Figs/logo.pdf}{0.8}

  \note{}
\end{frame}


\begin{frame}{}
\lfigh{Figs/logo_fundamentals.pdf}{0.2}

  \note{}
\end{frame}


\begin{frame}{}
\lfigh{Figs/logo_verify.pdf}{0.2}

  \note{}
\end{frame}


\begin{frame}{}
\lfigh{Figs/logo_explore.pdf}{0.2}

  \note{}
\end{frame}


\begin{frame}{}
\lfigh{Figs/logo_confer.pdf}{0.2}

  \note{}
\end{frame}


\begin{frame}{}
\lfigh{Figs/logo_document.pdf}{0.2}

  \note{}
\end{frame}


\begin{frame}[c]{}

\Large

Slides: \href{https://kbroman.org/Talk_DataCleaning}{\tt kbroman.org/Talk\_DataCleaning}

\vspace{7mm}

\href{https://kbroman.org}{\tt \lolit kbroman.org}

\vspace{7mm}

\href{https://github.com/kbroman}{\tt \lolit github.com/kbroman}

\vspace{7mm}

\href{https://twitter.com/kwbroman}{\tt \lolit @kwbroman}

\note{
}

\end{frame}




\end{document}
