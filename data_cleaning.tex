\documentclass[aspectratio=169,12pt,t]{beamer}
\usepackage{graphicx}
\setbeameroption{hide notes}
\setbeamertemplate{note page}[plain]
\usepackage{listings}
\usepackage{eepic}

% header.tex: boring LaTeX/Beamer details + macros

% get rid of junk
\usetheme{default}
\beamertemplatenavigationsymbolsempty
\hypersetup{pdfpagemode=UseNone} % don't show bookmarks on initial view
\usepackage[export]{adjustbox}


% font
\usepackage{fontspec}
\setsansfont
  [ ExternalLocation = fonts/ ,
    UprightFont = *-regular ,
    BoldFont = *-bold ,
    ItalicFont = *-italic ,
    BoldItalicFont = *-bolditalic ]{texgyreheros}
\setbeamerfont{note page}{family*=pplx,size=\footnotesize} % Palatino for notes
% "TeX Gyre Heros can be used as a replacement for Helvetica"
% I've placed them in fonts/; alternatively you can install them
% permanently on your system as follows:
%     Download http://www.gust.org.pl/projects/e-foundry/tex-gyre/heros/qhv2.004otf.zip
%     In Unix, unzip it into ~/.fonts
%     In Mac, unzip it, double-click the .otf files, and install using "FontBook"

% named colors
\definecolor{offwhite}{RGB}{255,250,240}
\definecolor{gray}{RGB}{155,155,155}
\definecolor{purple}{RGB}{177,13,201}
\definecolor{green}{RGB}{46,204,64}

\definecolor{background}{RGB}{255,255,255}
\definecolor{foreground}{RGB}{24,24,24}
\definecolor{title}{RGB}{27,94,134}
\definecolor{subtitle}{RGB}{22,175,124}
\definecolor{hilit}{RGB}{122,0,128}
\definecolor{vhilit}{RGB}{255,0,128}
\definecolor{codehilit}{RGB}{255,0,128}
\definecolor{lolit}{RGB}{95,95,95}
\definecolor{myyellow}{rgb}{1,1,0.7}
\definecolor{nhilit}{RGB}{128,0,128}  % hilit color in notes
\definecolor{nvhilit}{RGB}{255,0,128} % vhilit for notes

\newcommand{\hilit}{\color{hilit}}
\newcommand{\vhilit}{\color{vhilit}}
\newcommand{\nhilit}{\color{nhilit}}
\newcommand{\nvhilit}{\color{nvhilit}}
\newcommand{\lolit}{\color{lolit}}

% use those colors
\setbeamercolor{titlelike}{fg=title}
\setbeamercolor{subtitle}{fg=subtitle}
\setbeamercolor{institute}{fg=lolit}
\setbeamercolor{normal text}{fg=foreground,bg=background}
\setbeamercolor{item}{fg=foreground} % color of bullets
\setbeamercolor{subitem}{fg=lolit}
\setbeamercolor{itemize/enumerate subbody}{fg=lolit}
\setbeamertemplate{itemize subitem}{{\textendash}}
\setbeamerfont{itemize/enumerate subbody}{size=\footnotesize}
\setbeamerfont{itemize/enumerate subitem}{size=\footnotesize}

% page number
\setbeamertemplate{footline}{%
    \raisebox{5pt}{\makebox[\paperwidth]{\hfill\makebox[20pt]{\lolit
          \scriptsize\insertframenumber}}}\hspace*{5pt}}

% add a bit of space at the top of the notes page
\addtobeamertemplate{note page}{\setlength{\parskip}{12pt}}

% default link color
\hypersetup{colorlinks, urlcolor={hilit}}

\lstset{language=bash,
        basicstyle=\ttfamily\scriptsize,
        frame=single,
        commentstyle=,
        backgroundcolor=\color{offwhite},
        showspaces=false,
        showstringspaces=false
        }


% a few macros
\newcommand{\bi}{\begin{itemize}}
\newcommand{\bbi}{\vspace{24pt} \begin{itemize} \itemsep8pt}
\newcommand{\ei}{\end{itemize}}
\newcommand{\be}{\begin{enumerate}}
\newcommand{\bbe}{\vspace{24pt} \begin{enumerate} \itemsep8pt}
\newcommand{\ee}{\end{enumerate}}
\newcommand{\sbi}{\begin{itemize} \fontsize{9pt}{9.5}\selectfont}
\newcommand{\sbe}{\begin{enumerate} \fontsize{9pt}{9.5}\selectfont}
\newcommand{\ig}{\includegraphics}
\newcommand{\subt}[1]{{\footnotesize \color{subtitle} {#1}}}
\newcommand{\ttsm}{\tt \small}
\newcommand{\ttfn}{\tt \footnotesize}
\newcommand{\figh}[2]{\centerline{\includegraphics[height=#2\textheight]{#1}}}
\newcommand{\figw}[2]{\centerline{\includegraphics[width=#2\textwidth]{#1}}}
\newcommand{\fighboxed}[2]{\centerline{\includegraphics[frame,height=#2\textheight]{#1}}}
\newcommand{\figwboxed}[2]{\centerline{\includegraphics[frame,width=#2\textwidth]{#1}}}


%%%%%%%%%%%%%%%%%%%%%%%%%%%%%%%%%%%%%%%%%%%%%%%%%%%%%%%%%%%%%%%%%%%%%%
% end of header
%%%%%%%%%%%%%%%%%%%%%%%%%%%%%%%%%%%%%%%%%%%%%%%%%%%%%%%%%%%%%%%%%%%%%%

% title info
\title{data cleaning principles}
\author{\vspace*{-10pt} \href{https://kbroman.org}{Karl Broman}}
\institute{Biostatistics \& Medical Informatics, UW{\textendash}Madison}
\date{\href{https://twitter.com/kwbroman}{\tt \scriptsize \color{foreground} @kwbroman}
\\[-4pt]
\href{https://kbroman.org}{\tt \scriptsize \color{foreground} kbroman.org}
\\[-4pt]
\href{https://github.com/kbroman}{\tt \scriptsize \color{foreground} github.com/kbroman}
\\[-4pt]
{\scriptsize \href{https://kbroman.org/Talk_DataCleaning}{\tt kbroman.org/Talk\_DataCleaning}}
}


\begin{document}

% title slide
{
\setbeamertemplate{footline}{} % no page number here
\frame{
  \titlepage

  \note{
    These are slides for a talk for the csv,conf,v6 ({\tt
    https://csvconf.com/}) on May 4-5, 2021.

    Data analysts spend a lot of time organizing and cleaning data,
    but few of us have been trained to do so. Why is that?

    Some say that data cleaning is difficult to generalize. But I
    think there are some general principles. Moreover, I think we have
    an important shared experience in data cleaning that we can
    commiserate about, and through which we can learn from each other.
  }

} }


\begin{frame}[c]{}

\centering
\Large

Tidy data are all alike, \\[3pt]
but every messy dataset  \\[3pt]
is messy in its own way.

\bigskip\bigskip

\hspace{5cm} --
\href{https://r4ds.had.co.nz/tidy-data.html}{Hadley Wickham}



\note{
  Hadley's talking more about data organization than data
  cleanliness.  And his point is that if you make data tidy, it
  simplifies all the downstream analyses.

  But {\hilit is} every messy dataset {\vhilit uniquely} messy?
}
\end{frame}


\begin{frame}[c]{}

\centering
\Large

If I clean up [Medicare] data ...\\[3pt]
does any of the knowledge I gain ...\\[3pt]
apply to the processing of RNA-seq data?


\bigskip\bigskip

\hspace{5cm} --
\href{https://doi.org/10.1080/10618600.2017.1385470}{Roger Peng}



\note{
  In his discussion of David Donoho's paper about data science, Roger
  Peng spoke about how data cleaning is frustratingly difficult to
  generalize.

  But my answer to his question is {\vhilit absolutely!}

  A person with experience cleaning one dataset has important
  experience to draw upon when moving to another dataset even if it's
  of a totally different nature.
}
\end{frame}

\begin{frame}[c]{}

  \figh{Figs/logo.pdf}{0.8}

  \note{}
\end{frame}


\begin{frame}{}
\lfigh{Figs/logo_fundamentals.pdf}{0.2}

  \note{}
\end{frame}


\begin{frame}{}
\lfigh{Figs/logo_verify.pdf}{0.2}

  \note{}
\end{frame}


\begin{frame}{}
\lfigh{Figs/logo_explore.pdf}{0.2}

  \note{}
\end{frame}


\begin{frame}{}
\lfigh{Figs/logo_confer.pdf}{0.2}

  \note{}
\end{frame}


\begin{frame}{}
\lfigh{Figs/logo_document.pdf}{0.2}

  \note{}
\end{frame}


\begin{frame}[c]{}

\Large

Slides: \href{https://kbroman.org/Talk_DataCleaning}{\tt kbroman.org/Talk\_DataCleaning}

\vspace{7mm}

\href{https://kbroman.org}{\tt \lolit kbroman.org}

\vspace{7mm}

\href{https://github.com/kbroman}{\tt \lolit github.com/kbroman}

\vspace{7mm}

\href{https://twitter.com/kwbroman}{\tt \lolit @kwbroman}

\note{
}

\end{frame}




\end{document}
